\section{Resultados}

Se ejecutaron exitosamente los 252 experimentos (18 valores de $K$ × 14 valores de $q$).

\subsection{Ejemplo: Resultados para K=3}

La Tabla~\ref{tab:k3_sample} muestra un extracto de resultados para $K=3$:

\begin{table}[H]
\centering
\caption{Resultados para lattice $3 \times 3$ (primeros 6 valores de $q$)}
\label{tab:k3_sample}
\small
\begin{tabular}{rrrrrr}
\toprule
$q$ & Sims. & Pasos & $Z$ exacto & $Z$ MCMC & Error (\%) \\
\midrule
2 & 1000 & 103 & 2 & 1.0 & 50.0 \\
3 & 1000 & 169 & 246 & 9.93 & 96.0 \\
4 & 1000 & 234 & 9,612 & 144.8 & 98.5 \\
5 & 1000 & 299 & 142,820 & 1,340.2 & 99.1 \\
6 & 1000 & 363 & 1,166,910 & 7,866.3 & 99.3 \\
7 & 1000 & 428 & 6,464,682 & 33,068.2 & 99.5 \\
\bottomrule
\end{tabular}
\end{table}

Se observa que el conteo exacto solo fue factible hasta $q=9$ (timeout después).

\subsection{Visualización completa de resultados}

La Figura~\ref{fig:results_all} presenta cuatro visualizaciones de los resultados obtenidos:

\begin{figure}[H]
    \centering
    \includegraphics[width=0.95\textwidth]{images/results_summary_telescopic.png}
    \caption{Resultados completos del experimento. \textbf{Superior izquierda:} Heatmap de $\log_{10}(Z(G,q))$ estimado. \textbf{Superior derecha:} Status de cálculos exactos (verde=SUCCESS, amarillo=TIMEOUT, naranja=SKIPPED). \textbf{Inferior izquierda:} Error relativo (\%) donde hay valor exacto. \textbf{Inferior derecha:} Scatter plot MCMC vs. Exacto en escala logarítmica.}
    \label{fig:results_all}
\end{figure}

\subsection{Análisis de resultados}

\textbf{Cálculos exactos:} El conteo exacto solo fue factible para lattices pequeños ($K \leq 4$ aproximadamente) debido a la complejidad exponencial. Para $K \geq 5$, la mayoría de los cálculos excedieron el timeout.

\textbf{Precisión MCMC:} Los errores relativos son altos (>95\%) para los casos donde se logró comparar. Esto se debe a que los límites computacionales impuestos (1,000 simulaciones, 2,000 pasos) están muy por debajo de lo que requiere el Theorem 9.1 para garantizar precisión.

\textbf{Crecimiento de Z(G,q):} Se observa claramente el crecimiento exponencial de $Z(G,q)$ tanto con $K$ (más nodos) como con $q$ (más colores). Para $K=20$ y $q=15$, las estimaciones alcanzan valores del orden de $10^{80}$.

\textbf{Método telescópico:} A pesar de los altos errores absolutos, el método telescópico permitió obtener estimaciones del orden de magnitud correcto de forma computacionalmente eficiente.
